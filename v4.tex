%! TEX root = /home/simon/Documents/Anställning MV/TMV137/main.tex
\section{Vecka 47}%
\label{sec:vecka_47}
% 3.7:   6, 15, 17.      18.6:   2, 8.
%  9.1    21.    9.2   7, 21, 31.

\subsection{3.7 6}%
\label{sub:3_7_6}

Hitta alla lösningar till
\begin{align*}
	y'' - 2 y' + y = 0.
\end{align*}

\paragraph{Lösning:}

Vi har fall II (s.\ 207 i Adams), så den generella lösningen är på formen
\begin{align*}
	y = A \exp{r t} + B t \exp{r t}
\end{align*}
där $r$ är den dubbla roten till polynomet
\begin{align*}
	x^2 - 2 x + x = 0.
\end{align*}
Vi har att $r = 1$.


\subsection{3.7 15}%
\label{sub:3_7_15}

Lös begynnelsevärdesproblemet
\begin{align}
	y'' + 4 y + 5 y ={}& 0,%
	\label{eq:3_7_15_diffeq}\\
	y(0) ={}& 2,%
	\label{eq:3_7_15_boundcon1}\\
	y'(0) ={}& 2.%
	\label{eq:3_7_15_boundcon2}
\end{align}

\paragraph{Lösning:}

Vi har nu fall III ($b^2 - 4 a c < 0$).
Därför är den mest generella komplexa lösningen till \cref{eq:3_7_15_diffeq}
\begin{align*}
	y = A \exp{k t + i \omega t} + B \exp{k t - i \omega t}
\end{align*}
där $k = -b / (2 a) = -4 / 2 = -2$ och $\omega = \sqrt{4 a c - b^2} / (2 a) = \sqrt{4 \cdot 5 - 4^2} / 2 = 1$.
Begynnelsevillkoren \cref{eq:3_7_15_boundcon1,eq:3_7_15_boundcon2} $\implies$
\begin{align*}
	A + B ={}& 2,\\
	A (-2 + i) + B (-2 - i) ={}& 2.
\end{align*}
Vi får att
\begin{align*}
	B ={}& 2 - A,\\
	A (-2 + i) + (2 - A) (-2 - i) ={}&\\
	2 i A - 4 - 2 i ={}& 2\\
	\implies&\\
	A ={}& 1 - 3 i.\\
\end{align*}
Alltså är lösningen
\begin{align*}
	y ={}& (1 - 3 i) \exp{-2 t + i t} + (1 + 3 i) \exp{-2 t - i t}\\
	={}& 2 \Re \left\{ \exp{-2 t + i t} \right\} + 2 \Im \left\{3 \exp{-2 t + i t} \right\}\\
	={}& 2 \exp{-2 t} \cos{t} + 6 \exp{-2 t} \sin{t}
\end{align*}
Eftersom all koefficienter i diffekvationen och alla begynnelsevillkoren var rella är det betryggande att se att även lösningen är rell.


\subsection{3.7 17}%
\label{sub:3_7_17}

Visa att om $a$, $b$, $c > 0$ i diffekvationen
\begin{align*}
	a y'' + b y' + c = 0
\end{align*}
så gäller $\lim_{t \to \infty} y(t) = 0$ för alla lösningar.

\paragraph{Lösning:}

Vi har då att alla rötter till den karaktäristiska ekvationen
\begin{align*}
	a r^2 + b r + c = 0
\end{align*}
är negativa.
I fall I beskrivs den mest generella lösningen av
\begin{align*}
	y = A \exp{r_1 t} + B \exp{r_2 t} \overset{t \to \infty}{\longrightarrow} 0.
\end{align*}
I fall II beskrivs den mest generella lösningen av
\begin{align*}
	y = A \exp{r t} + B \exp{r t} \overset{t \to \infty}{\longrightarrow} 0.
\end{align*}
I fall III beskrivs den mest generella lösningen av
\begin{align*}
	y = A \exp{k t} \cos{\omega t} + B \exp{k t} \sin{\omega t},
\end{align*}
som också går mot noll när $t \to \infty$ eftersom $k = - b / (2 a) < 0$.


\subsection{18.6 2}%
\label{sub:18_6_2}

Hitta den mest generella lösningen till
\begin{align*}
	y'' + y' - 2 y = x.
\end{align*}

\paragraph{Lösning:}

Homogenlösningen ges av fall I:
\begin{align*}
	y = A \exp{r_1 t} + B \exp{r_2 t}
\end{align*}
där $r_1$ och $r_2$ är rötterna till $r^2 + r - 2 = 0$.

En partikulärlösning är
\begin{align}
	y = -\frac{1}{2} x + \frac{1}{4}.
\end{align}


\subsection{18.6 8}%
\label{sub:18_6_8}

Hitta den mest generella lösningen till
\begin{align*}
	y'' + 4 y' + 4 y = \exp{-2 t}.
\end{align*}

\paragraph{Lösning:}

Homogenlösningen ges av fall II:
\begin{align*}
	A \exp{r t} + B t \exp{r t}
\end{align*}
där $r$ är roten till $r^2 + 4 r + 4 = 0$.

En partikulärlösning är
\begin{align*}
	y = \frac{1}{2} t^2 \exp{-2 t}.
\end{align*}


\subsection{9.1 21}%
\label{sub:9_1_21}

Beräkna gränsvärdet av
\begin{align*}
	\left( \frac{n - 3}{n} \right)^n.
\end{align*}
när $n \to \infty$.

\paragraph{Lösning:}

\begin{theorem}[sats 6 i avsnitt 3.4 i Adams]
	\begin{align*}
		\exp{x} = \lim_{n \to \infty} \left( 1 + \frac{x}{n} \right)^n.
	\end{align*}
\end{theorem}

Vi har att
\begin{align*}
	\left( \frac{n - 3}{n} \right)^n ={}& \left( 1 + \frac{-3}{n} \right)^n,
\end{align*}
så gränsvärdet är $\exp{-3}$.


\subsection{9.2 7}%
\label{sub:9_2_7}

Hitta gränsvärdet eller visa att serien divergerar:
\begin{align*}
	\sum_{k = 0}^\infty \frac{2^{k + 3}}{\exp{k - 3}}.
\end{align*}

\paragraph{Lösning:}

\begin{align*}
	\sum_{k = 0}^\infty \frac{2^{k + 3}}{\exp{k - 3}} ={}& \sum_{k = 0}^\infty \exp{\log{2} (k + 3) - (k - 3)}\\
	={}& \exp{3 (\log{2} + 1)} \sum_{k = 0}^\infty \left( \exp{\log{2} - 1} \right)^k\\
	={}& (2 \exp{})^3 \sum_{k = 0}^\infty \left( 2 / \exp{} \right)^k\\
	={}& (2 \exp{})^3 \frac{1}{1 - 2 / \exp{}}\\
	={}& \frac{8 \exp{4}}{\exp{} - 2}\\
\end{align*}


\subsection{9.2 21}%
\label{sub:9_2_21}

En studsboll tappar $3/4$ av sin höjd varje studs.
Om den släpps från höjden $a$ och låts studsa i all oändlighet, vad är gränsvärdet av sträckan som bollen färdas genom luften?

\paragraph{Lösning:}

\begin{align*}
	\sum_{k = 0}^\infty a \left( \frac{3}{4} \right)^k ={}& \frac{a}{1 - \frac{3}{4}}\\
	={}& 4 a.
\end{align*}


\subsection{9.2 31}%
\label{sub:9_2_31}

Visa att påståendet är sant eller ge ett motexempel:
\begin{quote}
	$a_n > 0$ för alla $n$ och $\sum_n a_n$ konvergent $\implies$ $\sum_n (a_n)^2$ konvergent.
\end{quote}


\paragraph{Lösning:}

Låt $a = \max_n a_n$.
\begin{lemma}
	$a$ är väldefinerat.
\end{lemma}
\begin{proof}
	$\lim_{n \to \infty} a_n = 0$ enligt sats 4 från avsnitt 9.2, så för något $N$ är $a_n < a_0$ för alla $n > N$.
	Alltså har mängden $\set{a_0, \dots}$ ett max omm delmängden $\set{a_0, \dots, a_N}$ har ett max, men detta är en ändlig mängd så den har uppenbarligen ett max.
\end{proof}

Vi har då att
\begin{align}
	\sum_n (a_n)^2 ={}& a^2 \sum_n \left( \frac{a_n}{a} \right)^n\\
	\leq{}& a^2 \sum_n \left( \frac{a_n}{a} \right)\\
	\leq{}& a \sum_n a_n
\end{align}
som vi vet konvergerar :).









