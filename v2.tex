%! TEX root = /home/simon/Documents/Anställning MV/TMV137/main.tex
\section{Vecka 45}%
\label{sec:vecka_45}


\subsection{5.6 6}%
\label{sub:5_6_6}

Evaluera
\begin{align*}
	\int \frac{\sin \sqrt{x}}{\sqrt{x}} \dd{x}.
\end{align*}

\paragraph{Lösning:}

Kedjeregeln ger att
\begin{align*}
	\derivative{}{x} \cos \sqrt{x} = -\frac{\sin \sqrt{x}}{2 \sqrt{x}}.
\end{align*}
Vi har därför att
\begin{align*}
	\int \frac{\sin \sqrt{x}}{\sqrt{x}} \dd{x} = -2 \cos{\sqrt{x}} + C.
\end{align*}


\subsection{5.6 9}%
\label{sub:5_6_9}

Evaluera
\begin{align*}
	\int \frac{\cos{x}}{4 + \sin^2{x}} \dd{x}.
\end{align*}

\paragraph{Lösning:}

I Adams står det att
\begin{align*}
	\int \frac{1}{a^2 + x^2} \dd{x} = \frac{1}{a} \arctan \frac{x}{a} + C.
\end{align*}

Vi börjar med att göra variabelsubstitutionen $u = \sin x$.
Eftersom $\dd{u} = \cos{x} \dd{x}$ har vi att
\begin{align*}
	\int \frac{\cos{x}}{4 + \sin^2{x}} \dd{x} ={}& \int \frac{1}{4 + u^2} \dd{u}\\
	={}& \frac{1}{2} \arctan \frac{u}{2} + C\\
	={}& \frac{1}{2} \arctan \frac{\sin x}{2} + C.
\end{align*}


\subsection{5.7 10}%
\label{sub:5_7_10}

Hitta arean som begränsas av kurvorna
\begin{align*}
	x ={}& y^2,\\
	x ={}& 2 y^2 - y - 2.
\end{align*}

\paragraph{Lösning:}

Kurvorna möts i $y = -1$ och $y = 2$.
Alltså är den inneslutna arean
\begin{align*}
	\int_{-1}^{2} y^2 - (2 y^2 - y - 2) \dd{y} ={}& \int_{-1}^{2} -y^2 + y + 2 \dd{y}\\
	={}& \frac{9}{2}.
\end{align*}


\subsection{6.1 2}%
\label{sub:6_1_2}

Beräkna
\begin{align*}
	\int (x + 3) \exp{2 x} \dd{x}.
\end{align*}

\paragraph{Lösning:}

\begin{theorem}[partialintegrering]
	Om $u$ och $v$ är två deriverbara funktioner av $x$ så gäller
	\begin{align*}
		\int u \derivative{v}{x} \dd{x} = u v - \int \derivative{u}{x} v \dd{x}.
	\end{align*}
\end{theorem}

Låt $u = x + 3$ och $\derivative{v}{x} = \exp{2 x}$ ($v = \frac{1}{2} \exp{2 x}$).
Då har vi att
\begin{align*}
	\int (x + 3) \exp{2 x} \dd{x} ={}& (x + 3) \frac{1}{2} \exp{2 x} - \int \frac{1}{2} \exp{2 x} \dd{x}\\
	={}& \left((x + 3) \frac{1}{2} - \frac{1}{4}\right) \exp{2 x} + C.
\end{align*}


\subsection{6.1 7}%
\label{sub:6_1_7}

Beräkna
\begin{align*}
	\int \arctan x \dd{x}.
\end{align*}

\paragraph{Lösning:}

Vi vill använda att
\begin{align*}
	\derivative{}{x} \arctan{x} = \frac{1}{1 + x^2}.
\end{align*}
Så låt $u = \arctan{x}$ och $\derivative{v}{x} = 1$ ($v = x$).
Partiell integrering ger då att
\begin{align*}
	\int \arctan x \dd{x} ={}& x \arctan{x} - \int x \frac{1}{1 + x^2} \dd{x}\\
	={}& x \arctan{x} - \frac{1}{2} \log{(1 + x^2)} + C.
\end{align*}


\subsection{6.1 21}%
\label{sub:6_1_21}

Beräkna
\begin{align*}
	\int \frac{\log{\log{x}}}{x} \dd{x}.
\end{align*}


\paragraph{Lösning:}

Låt $u = \log{x}$.
Då är $\dd{u} = \frac{1}{x} \dd{x}$ och
\begin{align*}
	\int \frac{\log{\log{x}}}{x} \dd{x} ={}& \int \log{u} \dd{u}\\
	={}& u \log{u} - \int 1 \dd{u}\\
	={}& u \log{u} - u + C.
\end{align*}


\subsection{6.2 7}%
\label{sub:6_2_7}

Beräkna
\begin{align*}
	\int \frac{1}{a^2 - x^2} \dd{x}.
\end{align*}

\paragraph{Lösning:}

Vi använder oss av partialintegrering:
\begin{align*}
	\frac{1}{a^2 - x^2} ={}& \frac{1}{(a - x) (a + x)}\\
	={}& \frac{A}{a - x} + \frac{B}{a + x}
\end{align*}
för några konstanter $A$ och $B$.
Vi kan bestämma dom genom att vi vet att
\begin{align*}
	A (a + x) + B (a - x) = 1,
\end{align*}
så $A = B = \frac{1}{2 a}$.
Nu är det lätt att integrera:
\begin{align*}
	\int \frac{1}{a^2 - x^2} \dd{x} ={}& \int \frac{A}{a - x} + \frac{B}{a + x} \dd{x}\\
	={}& -A \log{(a - x)} + B \log{(a + x)} + C\\
	={}& \frac{1}{2 a} \left(\log{(a + x)} - \log{(a - x)}\right) + C.
\end{align*}


\subsection{6.2 22}%
\label{sub:6_2_22}

Beräkna
\begin{align*}
	\int \frac{x^2 + 1}{x^3 + 8} \dd{x}.
\end{align*}

\paragraph{Lösning:}

Vi har att
\begin{align*}
	\frac{x^2 + 1}{x^3 + 8} ={}& \frac{x^2}{x^3 + 8} + \frac{1}{(x + 2) (x^2 - 2 x + 4)} 
\end{align*}
Eftersom $x^3 + 8 = (x + 2) (x^2 - 2 x + 4)$ kan vi dela upp
\begin{align*}
	\frac{1}{(x + 2) (x^2 - 2 x + 4)} ={}& \frac{2}{12 (x^2 - 2 x + 4)} - \frac{x - 2}{12 (x^2 - 2 x + 4)} + \frac{1}{12 (x + 2)}.
\end{align*}
Vi vet hur man integrerar varje term:
\begin{align*}
	\int \frac{x^2}{x^3 + 8} \dd{x} \overset{u = x^3}&{=} \int \frac{1}{3 (u + 8)} \dd{u}\\
	&= \frac{1}{3} \log{\abs{u + 8}} + C\\
	&= \frac{1}{3} \log{\abs{x^3 + 8}} + C,\\
	%
	\int \frac{x - 2}{x^2 - 2 x + 4} \dd{x} \overset{u = x^2 - 2 x}&{=} \int \frac{1}{2 (u + 4)} \dd{x}\\
	&= \frac{1}{2} \log{\abs{u + 4}} + C\\
	&= \frac{1}{2} \log{\abs{x^2 - 2 x + 4}} + C,\\
	%
	\int \frac{2}{x^2 - 2 x + 4} \dd{x} \overset{u + 1 = x}&{=} \int \frac{2}{u^2 + 3} \dd{x}\\
	&= \frac{2}{\sqrt{3}} \arctan{\frac{u}{\sqrt{3}}} + C\\
	&= \frac{2}{\sqrt{3}} \arctan{\frac{x - 1}{\sqrt{3}}} + C,\\
	%
	\int \frac{1}{x + 2} \dd{x} &= \log{\abs{x + 2}} + C.
\end{align*}
Vi kombinerar dessa fyra identiteter för att erhålla
\begin{align*}
	\int \frac{x^2 + 1}{x^3 + 8} \dd{x} ={}& \int \frac{x^2}{x^3 + 8} + \frac{1}{(x + 2) (x^2 - 2 x + 4)} \dd{x}\\
	={}& \frac{1}{3} \log{\abs{x^3 + 8}} + \frac{1}{12} \left( \frac{2}{\sqrt{3}} \arctan{\frac{x - 1}{\sqrt{3}}} - \frac{1}{2} \log{\abs{x^2 - 2 x + 4}} + \log{\abs{x + 2}} \right) + C.
\end{align*}


\subsection{6.5 8}%
\label{sub:6_5_8}

Beräkna
\begin{align*}
	\int_0^1 \frac{1}{x \sqrt{1 - x}} \dd{x}
\end{align*}
eller bevisa att den divergerar.

\paragraph{Lösning:}

Integranden kommer bete sig som $1 / x$ runt $x = 0$, så vi gissar att integralen kommer divergera eftersom
\begin{align*}
	\int_0^p \frac{1}{x} \dd{x} ={}& \lim_{\epsilon \to 0} \int_\epsilon^p \frac{1}{x} \dd{x}\\
	={}& \lim_{\epsilon \to 0} [\log{x}]_\epsilon^p\\
	={}& \lim_{\epsilon \to 0} \log{p} - \log{\epsilon}\\
	={}& \infty.
\end{align*}
för alla $p > 0$.
Så vi vill visa att integralen divergerar.

Integranden är alltid positiv, så
\begin{align*}
	\int_0^1 \frac{1}{x \sqrt{1 - x}} \dd{x} \geq{}& \int_0^\frac{1}{2} \frac{1}{x \sqrt{1 - x}} \dd{x}\\
	\geq{}& \int_0^\frac{1}{2} \frac{1}{x \sqrt{1 / 2}} \dd{x},
\end{align*}
vilket vi har visat divergerar mot oändligheten.


\subsection{6.5 19}%
\label{sub:6_5_19}

Beräkna
\begin{align*}
	\int_{-\infty}^\infty \frac{x}{1 + x^2} \dd{x}
\end{align*}
eller bevisa att den divergerar.

\paragraph{Lösning:}

Detta är en udda funktion integrerad över ett symmetriskt intervall, så integralen är $0$ om den existerar.
Dock är
\begin{align*}
	\int_{0}^\infty \frac{x}{1 + x^2} \dd{x} ={}& \lim_{M \to \infty} [\frac{1}{2} \log{(1 + x^2)}]_0^M\\
	={}& \lim_{M \to \infty} \frac{1}{2} \log{(1 + M^2)}\\
	={}& \infty
\end{align*}
så integralen existerar inte.


\subsection{6.5 34}%
\label{sub:6_5_34}

Divergerar eller konvergerar
\begin{align*}
	\int_0^\infty \frac{\dd{x}}{\sqrt{x} + x^2}?
\end{align*}

\paragraph{Lösning:}

Konvergerar pga
\begin{align*}
	\int_0^\infty \frac{\dd{x}}{\sqrt{x} + x^2}
	={}& \int_0^1 \frac{\dd{x}}{\sqrt{x} + x^2}
	+ \int_1^\infty \frac{\dd{x}}{\sqrt{x} + x^2}\\
	%
	\leq{}& \int_0^1 \frac{\dd{x}}{2 \sqrt{x}}
	+ \int_1^\infty \frac{\dd{x}}{2 x^2} \\
	<{}& \infty.
\end{align*}


\subsection{6.5 35}%
\label{sub:6_5_35}

Divergerar eller konvergerar
\begin{align*}
	\int_{-1}^{1} \frac{\exp{x}}{x + 1} \dd{x}?
\end{align*}

\paragraph{Lösning:}

Divergerar pga
\begin{align*}
	\int_{-1}^{1} \frac{\exp{x}}{x + 1} \dd{x} \geq{}& \int_{-1}^{1} \frac{\exp{-1}}{x + 1} \dd{x}\\
	={}& \int_{0}^{2} \frac{\exp{-1}}{x} \dd{x}\\
	={}& \infty.
\end{align*}


\subsection{6.5 42a}%
\label{sub:6_5_42a}

Givet att
\begin{align*}
	\int_0^\infty \exp{-x^2} \dd{x} = \frac{1}{2} \sqrt{\pi},
\end{align*}
beräkna
\begin{align*}
	\int_0^\infty x^2 \exp{-x^2} \dd{x}.
\end{align*}

\paragraph{Lösning}

Vi kör lite partialintegrering med $u = x$, $\derivative{v}{x} = x \exp{-x^2}$ ($v = -\frac{1}{2} \exp{-x^2}$):
\begin{align*}
	\int_0^\infty x^2 \exp{-x^2} \dd{x} ={}& [-\frac{x}{2} \exp{-x^2}]_0^\infty + \int_0^\infty \frac{1}{2} \exp{-x^2} \dd{x}\\
	={}& \frac{1}{4} \sqrt{\pi}.
\end{align*}


