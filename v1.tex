
\subsection{5.1 19}%
\label{sub:5_1_19}

Hitta ett uttryck på sluten form för
\begin{align}
	\sum_{k = 1}^n (\pi^k - 3).
\end{align}

\paragraph{Lösning:}%
\label{par:losning_}

Vi vill använda följande sats:
\begin{theorem}[5.1 (d) i Adams]
	\begin{align}
		\sum_{k = 1}^n r^{k - 1} = \frac{r^n - 1}{r - 1}
	\end{align}
	om $r \neq 1$.
\end{theorem}
Vi får då att
\begin{align*}
	\sum_{k = 1}^n (\pi^k - 3) ={}& -3 n + \pi \sum_{k = 1}^n \pi^{k - 1}\\
	={}& -3 n + \pi \frac{\pi^n - 1}{\pi - 1}.
\end{align*}


\subsection{5.4 21}%
\label{sub:5_4_21}

Vi vet att
\begin{align}\label{eq:hint_in_5.4_19}
	\int_0^a x^2 \dd{x} = \frac{a^3}{3}.
\end{align}
Beräkna
\begin{align}\label{eq:integral_from_5.4_19}
	\int_0^1 (x^2 + \sqrt{1 - x^2}) \dd{x}.
\end{align}

\paragraph{Lösning:}

Den första termen i \cref{eq:integral_from_5.4_19} ser vi från \cref{eq:hint_in_5.4_19} är $\frac{1}{3}$.
Den andra termen kan vi identifiera med arean av en kvarts \todo[inline]{(rita figur)} och den blir $\frac{\pi}{4}$.


\subsection{5.4 36}%
\label{sub:5_4_36}

Beräkna
\begin{align*}
	\int_0^3 \abs{2 - x} \dd{x}.
\end{align*}

\paragraph{Lösning:}

\begin{align*}
	\int_0^3 \abs{2 - x} \dd{x} ={}& \int_0^2 (2 - x) \dd{x} + \int_2^3 -(2 - x) \dd{x}\\
	={}& 2 + \frac{1}{2}.
\end{align*}


\subsection{5.5 16}%
\label{sub:5_5_16}

Beräkna
\begin{align*}
	\int_{-1}^1 2^x \dd{x}.
\end{align*}

\paragraph{Lösning:}

\begin{align*}
	\int_{-1}^1 2^x \dd{x} ={}& \left[\log{2} \cdot 2^x\right]_{-1}^1\\
	={}& \log{2}(2 - \frac{1}{2})\\
	={}& \frac{3}{2} \log{2}.
\end{align*}


\subsection{5.5 25}%
\label{sub:5_5_25}

Hitta arean som begränsas av kurvorna
\begin{align*}
	y ={}& x^2 - 3 x + 3,\\
	y ={}& 1.
\end{align*}

\paragraph{Lösning}

\todo[inline]{Rita en bild.}\\
\todo[inline]{Snacka allmänt om arean av en figur som begränsas av två kurvor $y = f(x)$ och $y = g(x)$.}\\
Eftersom kurvorna möts i $x = 1$ och $x = 2$ är arean (beloppet av)
\begin{align*}
	\int_1^2 (x^2 - 3 x + 3 - 1) \dd{x} = -\frac{1}{6}.
\end{align*}

\subsection{5.5 44}%
\label{sub:5_5_44}

Beräkna
\begin{align*}
	\derivative{}{\theta} \int_{\sin \theta}^{\cos \theta} \frac{1}{1 - x^2} \dd{x}.
\end{align*}

\paragraph{Lösning}

\begin{align*}
	\int_{\sin \theta}^{\cos \theta} \frac{1}{1 - x^2} \dd{x} ={}&
	\int_{1}^{\cos \theta} \frac{1}{1 - x^2} \dd{x}
	+ \int_{\sin \theta}^{1} \frac{1}{1 - x^2} \dd{x}\\
	%
	={}& \int_{1}^{\cos \theta} \frac{1}{1 - x^2} \dd{x}
	- \int_{1}^{\sin \theta} \frac{1}{1 - x^2} \dd{x}.
\end{align*}
Enligt kedjeregeln är
\begin{align*}
	\derivative{}{\theta} \int_{1}^{\cos \theta} \frac{1}{1 - x^2} \dd{x} ={}&
	\derivative{}{\cos \theta} \left(\int_{1}^{\cos \theta} \frac{1}{1 - x^2} \dd{x}\right) \derivative{\cos \theta}{\theta}\\
	%
	={}& -\frac{1}{1 - \cos^2 \theta} \sin \theta\\
	%
	={}& -\sin \theta.
\end{align*}
och
\begin{align*}
	\derivative{}{\theta} \int_{1}^{\sin \theta} \frac{1}{1 - x^2} \dd{x} ={}&
	\derivative{}{\sin \theta} \left(\int_{1}^{\sin \theta} \frac{1}{1 - x^2} \dd{x}\right) \derivative{\sin \theta}{\theta}\\
	%
	={}& \frac{1}{1 - \sin^2 \theta} \cos \theta\\
	%
	={}& \cos \theta.
\end{align*}


\subsection{Chapter 5 review 21}%
\label{sub:chapter_5_review_21}

Beräkna arean som begränsas av
\begin{align*}
	y = \sin x,\\
	y = \cos 2 x,\\
	x = 0,\\
	x = \frac{\pi}{6}.
\end{align*}

\paragraph{Lösning:}

\todo[inline]{TODO.}

